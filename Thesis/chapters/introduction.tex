\subsection{Background}
A Vector Network Analyser is a radio frequency test and measurement instrument which measures the s-parameters of a device under test. VNAs are used extensively in the design of RF devices as they allow the gain and phase response of RF components such as antennas, filters, amplifiers, and mixers to be characterised over the their operating frequencies, which is crucial for successful design and manufacturing of RF devices. \par

Given that VNAs are intricate pieces of engineering and are sold in low volume to engineering firms, their price represents this high level of performance and niche market, with VNAs typically ranging from tens of thousands to millions of dollars. Given this pricing is prohibitively expensive for teaching in educational institutions and far more than most electronics hobbyists have to spend, this limits the scope of projects and learning experiences which can be accessed outside of professional engineering environments. 

\subsection{Aim}
The aim of this thesis is design and build a VNA which is significantly cheaper than the commercial offerings, as this will lower the barrier of entry to RF engineering and enable more hobbyists and educators to build or purchase a VNA which fits their needs. Whilst this VNA will not have the frequency and dynamic range, accuracy, UI, or speed of a commercial VNA, it will still be able to provide indicative measurements which are sufficiently accurate for home or educational use, along with providing an interesting learning exercise in the areas of radio frequency and embedded engineering.  

\subsection{Overview}

\textbf{Chapter 2 - Requirements, Constraints, and Specifications} details the requirements and constraints of the VNA, and will enable the design scope and device specifications to be determined. \par

\textbf{Chapter 3 - Design} covers design of the VNA. Resulting from theory of operation and research of other low cost VNAs, the high level architecture along with detailed design was undergone and though testing candidate components were validated and selected.  \par

\textbf{Chapter 4 - Implementation and Assembly} details printed circuit board layout along with assembly of the PCB. Furthermore, mechanical design of an enclosure along with firmware and software implementation is discussed. \par

\textbf{Chapter 5 - Calibration and Testing} covers calibration of the VNA, comparison of measurements to commercial VNAs, along with a discussion of the results and proposes future avenues for improvement of the VNAs. 